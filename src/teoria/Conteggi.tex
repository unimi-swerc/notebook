\subsubsection{Numeri di Stirling del primo tipo}
Numero di permutazioni di $n$ elementi con $k$ cicli. I punti fissi sono considerati come cicli di lunghezza 1.
$$ \stirling{n}{k} = \stirling{n-1}{k-1} + (n-1)\stirling{n-1}{k}, \ \stirling{0}{0}=1, \ \stirling{n}{0}=\stirling{0}{n}=0$$
%$$ c(n,k) = c(n-1,k-1) + (n-1) c(n-1,k),\ c(0,0) = 1, \ c(n,0)=c(0,n)=0$$

\subsubsection{Numeri di Stirling del secondo tipo}
Numero di modi per partizionare $n$ oggetti (diversi) in $k$ sottoinsiemi non vuoti.
$$ \Stirling{n}{k} = \Stirling{n-1}{k-1} + k\Stirling{n-1}{k}, \ \Stirling{0}{0}=1,\ \Stirling{n}{0}=\Stirling{0}{n}=0 $$

\subsubsection{Derangement}
Numero di permutazioni di $n$ elementi senza punti fissi: $$ D(n)=nD(n-1)+(-1)^n $$

\subsubsection{Numeri di Catalan}
Numero di alberi binari pieni (ogni nodo padre ha esattamente due figli) con $n$ nodi padre.
Numero di alberi (radicati) non isomorfi con $n+1$ nodi, in cui però conta la posizione dei figli (sottoalbero sx con 3 nodi e sottoalbero dx con 5 nodi è diverso da sottoalbero sx con 5 nodi e sottoalbero dx con 3 nodi).
Numero di parole di Dyck di lunghezza $2n$ e tasselazioni di una scala con $n$ gradini con $n$ rettangoli.
Numero di permutazioni $\{1,\dots,n\}$ ordinabili tramite stack (ovvero in cui non compaiono sottosequenze crescenti di lunghezza 3, cambiando ordinamento si può vietare un qualsiasi pattern da 3) e numero di matrici $2\times n$ con righe e colonne ordinate.
$$ C(n)=\binom{2n}{n}-\binom{2n}{n+1}=\frac{1}{n+1}\binom{2n}{n} $$

\subsubsection{Burnside's lemma}
		Given a group $G$ of symmetries and a set $X$, the number of elements of $X$ \emph{up to symmetry} equals
		 \[ {\frac {1}{|G|}}\sum _{{g\in G}}|X^{g}|, \]
		 where $X^{g}$ are the elements fixed by $g$ ($g.x = x$).

		 If $f(n)$ counts ``configurations'' (of some sort) of length $n$, we can ignore rotational symmetry using $G = \mathbb Z_n$ to get
		 \[ g(n) = \frac 1 n \sum_{k=0}^{n-1}{f(\text{gcd}(n, k))} = \frac 1 n \sum_{k|n}{f(k)\phi(n/k)}. \]

\subsubsection{Somma potenze quarte}
$$ 1^4+...+n^4=\frac{n(n+1)(2n+1)(3n^2+3n-1)}{30} $$

\subsubsection{Euler's formula} $F+V-E=2$ (only for convex polyhedron).
