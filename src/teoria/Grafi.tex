\subsubsection{Relazione generale tra max indipendent set e minimum vertex cover}
In un grafo qualsiasi, se si considera un qualsiasi max indipendent set $S$ allora l'insieme dei nodi che non appartengono a $S$ formano un minimum vertex cover. In generale il complemento di un indipendent set (non per forza massimo) è sempre un vertex cover (non per forza minimo). Vale anche nel caso pesato.

\subsubsection{Max-flow min-cut theorem}
Dato un grafo pesato diretto (se gli archi non sono direzionati, basta scomporre ciascun arco bidirezionale in due archi direzionati e tutto funziona) il flusso massimo tra un nodo $u$ e un nodo $v$ è uguale al taglio minimo, ovvero il costo minimo di archi da rimuovere (costo di rimozione = peso dell'arco) in modo tale da disconnettere $u$ da $v$. Per trovare il taglio minimo quindi basta trovare il flusso massimo da $u$ a $v$, dopodichè fare una visita partendo da $u$ del grafo residuo e tutti gli archi del grafo originale che collegano un nodo visitato con un nodo non visitato fanno parte del taglio minimo.

\subsubsection{Konig Theorem: max matching e min vertex cover (anche pesato) su grafo bipartito}
In un grafo bipartito la cardinalità del max matching è uguale a quella del minimum vertex cover. Per trovare il max matching, creare una sorgente $u$ e 
degli archi di peso $1$ dalla sorgente verso tutti i nodi della prima metà del grafo bipartito, dopodichè per ogni arco $(a,b)$ del grafo bipartito creare un arco $a$->$b$ di peso $\infty$ 
(va bene anche di peso 1, ma se si tratta di min vertex cover pesato devono essere di peso infinito), infine creare un pozzo $v$ e degli archi di peso $1$ da tutti i nodi della 
seconda metà del grafo bipartito verso il pozzo. Il flusso massimo sarà uguale al max matching, inoltre il flusso massimo è uguale anche al taglio minimo e in questo caso il taglio 
minimo rappresenta il minimum vertex cover (per ogni arco del grafo bipartito originale, almeno uno dei due estremi deve essere preso/tagliato via da $u$ o $v$). Per min vertex cover pesato, modificare gli archi che partono dalla sorgente o arrivano verso il pozzo con il peso corrispettivo del nodo.

\subsubsection{Hall Theorem: matching perfetto su grafo bipartito}
Dato un insieme $W$ di nodi appartenenti al lato sinistro di un grafo bipartito $G$, sia $N_G(W)$ l'insieme dei vicini di $W$. $G$ ammette un matching perfetto per il lato sinistro (ovvero in cui ogni nodo del lato sinistro è matchato) se e solo se per ogni sottoinsieme $W$ (dei nodi del lato sinistro) vale $|W|\leq|N_G(W)|$.

\subsubsection{Dilworth Theorem e duale: insieme parzialmente ordinati, cammini ricoprenti e anti-catene}
\,

\textbf{Teorema}: in un insieme parzialmente ordinato, la cardinalità del più piccolo partizionamento in catene (le catene non devono avere elementi in comune, ma anche se li hanno 
non è un problema perchè essendo un ordinamento parziale si può modificare una catena  "saltando" alcuni oggetti facendo in modo che non abbia più elementi in comune con le altre) è
uguale alla cardinalità massima di una anti-catena. Una anti-catena è un insieme di elementi in cui nessuna coppia di elementi è confrontabile, una catena è un insieme di elementi 
tutte le coppie di elementi sono  confrontabili tra loro. 

\textbf{Duale}: la cardinalità massima di una catena è uguale alla cardinalità del più piccolo partizionamento in anti-catene (qua 
ovviamente non ha senso che due anti-catene abbiano elementi in comune, perchè se li hanno allora è possibile rimuovere tali elementi da uno dei due insiemi). 

\textbf{Trovare il partizionamento minimo in catene} (ogni elemento appartiene a una sola catena) in un insieme parzialmente ordinato: basta creare per ogni elemento $x$ un nodo $x'$ e $x''$, dopodichè creare un arco $x'$<->$y''$ se e solo se $x>y$ e trovare il max matching. Se
$x'$<->$y''$ appartiene al matching allora $x$ e $y$ sono nella stessa catena. 

\textbf{Trovare il numero minimo di cammini ricoprenti} (possono avere nodi in comune, altrimenti non si può fare) in un dag: trasformo il dag in un insieme parzialmente ordinato aggiungendo gli archi per creare la transitività e poi faccio come prima, ricordandomi di aggiustare le catene che ottengo perchè potrebbero fare dei "salti".

\textbf{Trovare la catena di cardinalità massima}: banale, è il cammino massimo in un dag. 

\textbf{Trovare l'anti-catena di cardinalità massima}: prima trova il partizionamento
minimo in catene, prendi tutti i nodi di partenza della catene e finchè c'è una coppia di nodi comparabili fai avanzare la testa del nodo più grande finchè sono tutti incomparabili (lca style).

\textbf{Trovare il partizionamento minimo in anti-catene}: TO DO. 

\subsubsection{Archi o nodi obbligati in un matching}
Dato un grafo residuo, un arco del max flow appartiene a tutti i max flow possibili se ha gli estremi in due componenti fortemente connesse diverse. Nel gioco su grafo bipartito in cui ogni nodo può venire visitato massimo una volta, i nodi che appartengono a tutti i max matching sono tutti e i soli nodi di partenza vincenti. Stesso gioco ma visitando ogni arco al massimo una volta: gaussin elimination per trovare l'insieme di archi su cui dirigere l'avversario.

\subsubsection{Max matching, in caso di parità scelgo min/max cost}
\,

\textbf{Min cost, grafo generale}: dare peso $\infty - w_i$ agli archi ($w_i$ è il costo dell'arco). Prima la priorità verrà data a trovare il max matching ovviamente, a parità di matching verrà scelto il costo minimo.

\textbf{Min cost, grafo bipartito}: max flow min cost.

\textbf{Per max cost}: uguale a min cost ma prima negare i $w_i$.

\subsubsection{Max matching pesato e min matching pesato}
\,

\textbf{Max matching, grafo generale}: weighted blossom algorithm. (se ci sono archi con peso negativo li butti perchè non ha senso prenderli)

\textbf{Max matching, grafo bipartito}: algoritmo ungherese, invertendo il valore dei $w_i$ ($w_i=-\infty$ se non esiste l'arco, poi negandolo diventerà $+\infty$) e aggiungendo eventuali nodi dummy (dalla parte dei lavoratori) che verrano matchati con i nodi che non appartengono al max match pesato.

\textbf{Min matching}: uguale a max matching ma pima negare i $w_i$.


\subsubsection{Min/Max cost ma non per forza max flow}
Np-completo

\subsubsection{Number of Spanning Trees}
		% I.e. matrix-tree theorem.
		% Source: https://en.wikipedia.org/wiki/Kirchhoff%27s_theorem
		% Test: stress-tests/graph/matrix-tree.cpp
		Create an $N\times N$ matrix \texttt{mat}, and for each edge $a \rightarrow b \in G$, do
		\texttt{mat[a][b]--, mat[b][b]++} (and \texttt{mat[b][a]--, mat[a][a]++} if $G$ is undirected).
		Remove the $i$th row and column and take the determinant; this yields the number of directed spanning trees rooted at $i$
		(if $G$ is undirected, remove any row/column). Oriented spanning tree = spanning tree (quindi n-1 archi) in cui dalla radice è possibile raggiungere ogni nodo

\subsubsection{Erdős–Gallai theorem}
		% Source: https://en.wikipedia.org/wiki/Erd%C5%91s%E2%80%93Gallai_theorem
		% Test: stress-tests/graph/erdos-gallai.cpp
		A simple graph with node degrees $d_1 \ge \dots \ge d_n$ exists iff $d_1 + \dots + d_n$ is even and for every $k = 1\dots n$,
		\[ \sum _{i=1}^{k}d_{i}\leq k(k-1)+\sum _{i=k+1}^{n}\min(d_{i},k). \]

\subsubsection{BEST theorem: number of eulerian circuits}
The BEST theorem states that the number ec$(G)$ of Eulerian circuits in a connected Eulerian graph $G$ is given by the formula:
$$\text{ec}(G)=t_w(G)\prod_{v\in V}(\text{deg}(v)-1)!$$
Here $t_w(G)$ is the number of arborescences, which are trees directed towards the root at a fixed vertex w in G (If we reverse the edges we obtain an oriented spanning tree, we can count the number of oriented spanning tree with a determinant as explained in the section above).
It is a property of Eulerian graphs that $t_v(G) = t_w(G)$ for every two vertices $v$ and $w$ in a connected Eulerian graph $G$. Counting eulerian circuit in a undirected graphs is at least as hard as NP-complete problem.

\,
