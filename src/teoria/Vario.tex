\subsubsection{Subset sum speedup for small weights}
There are $N$ items. The $i$-th item has weight $w_i\leq D$. Can we find a set $S$ such that $\sum_{i\in S}w_i=C$?
We will solve this in $\mathcal{O}(ND)$. Firstly, if $\sum w_i<C$, the answer is obviously no, so we will ignore that case.
Let us find the maximal $k$ such that $\sum_{i=1}^{k}w_i<C$. The basic idea is that we initially have an answer of $w_1+w_2+\dots+w_k$, then we can either subtract $w_i$ for $i\leq k$ or add $w_i$ for $i>k$ in some order such that the cost of our items is in the range $[C-D,C+D]$. The proof sketch is just: if the current weight is more than $C$, remove something, otherwise add something.
Let us define $\mathit{can}(\mathit{total},l,r)$ as a dp function that returns true iff there exists $\lambda_l,\lambda_{l+1},\dots,\lambda_r \in [0,1]$ such that $\sum_{i=1}^{l-1}w_i+\sum_{i=l}^{r}\lambda_i w_i=\mathit{total}$, where $\mathit{total}\in[C-D,C+D]$.
Notice that $\mathit{can}(\mathit{total},l,r)=\mathit{true}$ implies that $\mathit{can}(\mathit{total},l-1,r)=\mathit{true}$, this means that $\mathit{can}$ is monotone on the dimension $l$. Therefore, let us define a new dp function $\mathit{dp}(\mathit{total},r)$ that stores the maximal $l$ such that $\mathit{can}(\mathit{total},l,r)=\mathit{true}$.
Furthermore, can is monotone on the dimension $r$, so $\mathit{dp}(\mathit{total},r)\leq \mathit{dp}(\mathit{total},r+1)$.
Let us consider the transitions.
From $\mathit{dp}(\mathit{total},r)=l$, we can extend $r$, transitioning to $\mathit{dp}(\mathit{total}+w_{r+1},r+1)$ or $\mathit{dp}(\mathit{total},r+1)$. We can also extend $l$ and transition to $\mathit{dp}(\mathit{total}-w_{l'},r)=l'$ for $l'<l$. However, this transition would be $\mathcal{O}(N)$ per state, which is quite bad.
However, it would only make sense to transition to $\mathit{dp}(\mathit{total}-w_{l'},r)=l'$ for $\mathit{dp}(\mathit{total},r-1)\leq l'<\mathit{dp}(\mathit{total},r)=l$, otherwise this case would have been covered by another state and there would be no need for this transition. Since $\mathit{dp}(\mathit{total},r)\leq \mathit{dp}(\mathit{total},r+1)$, the total number of transitions by extending $l$ is actually bounded by $\mathcal{O}(ND)$ using amortized analysis.

\subsubsection{Varianti del Nim} 
\,

\textbf{Miserè nim}: chi fa l'ultima mossa perde. Per determinare il vincitore si procede come il nim normale (facendo lo xor dell'altezza di tutte le pile e si guarda se è zero), ma con una sola eccezione: se le altezze sono tutte 1, allora il vincitore è quello che nel nim normale perderebbe. Invece tutti gli altri casi è come il nim classico.

\textbf{K-nim}: ogni giocatore in un turno può diminuire l'altezza da $1$ pila fino a $k$ pile (se $k=1$ è il nim classico). Per determinare chi vince scrivere tutte le altezze delle pile in base 2, poi sommare questi numeri però supponendo che siano in base $k+1$ senza effetuare riporti (quindi come uno xor bit a bit ma modulo $k+1$ anzichè il classico modulo 2). Se il risultato è zero il giocatore di turno perde, altrimenti vince.

\subsubsection{Rango matrice}
Numero di linee (o colonne, è equivalente) linearmente indipendenti. Per calcolarlo: gaussian elimination (se ogni vettore ha al massimo 2 componenti non nulle si può velocizzare con dsu).

\subsubsection{Prodotto scalare (dot) e vettoriale (cross)} 
\,

\textbf{Dot product}: 
$$a\cdot b = |a| \cos \theta \cdot |b| = x_1 x_2 + y_1 y_2 + z_1 z_2$$

\textbf{Cross product}:  
$$|a\times b| = |a| \cdot |b| \sin \theta = (y_1 z_2 - z_1 y_2) + (z_1 x_2 - x_1 z_2) + (x_1 y_2 - y_1 x_2)$$ 
Verso e direzione: regola della mano destra, $\theta$ nel cross product 3d originale è sempre minore di 180 gradi, ma se si usa lo pseudo-scalar product in 2d $\theta$ è l'angolo da $a$ a $b$ in senso antiorario.

\subsubsection{Pick's Theorem}
Given a certain lattice polygon with non-zero area.
We denote its area by $S$, the number of points with integer coordinates lying strictly inside the polygon by $I$ and the number of points lying on polygon sides by $B$.
Pick's formula states: $S=I+\frac{B}{2}-1$.

\subsubsection{Euler's formula} $F+V-E=2$ (only for convex polyhedron).

\subsubsection{Minkowski sum}
Consider two sets $A$ and $B$ of points on a plane. Minkowski sum $A + B$ is defined as $\{a + b| a \in A, b \in B\}$.
Here we will consider the case when $A$ and $B$ consist of convex polygons $P$ and $Q$ with their interiors.
Here we consider the polygons to be cyclically enumerated, i. e. $P_{|P|} = P_0,\ Q_{|Q|} = Q_0$ and so on.

Suppose that both polygons are ordered counter-clockwise. Consider sequences of edges $\{\overrightarrow{P_iP_{i+1}}\}$
and $\{\overrightarrow{Q_jQ_{j+1}}\}$ ordered by polar angle. We claim that the sequence of edges of $P + Q$ can be obtained by merging
these two sequences preserving polar angle order and replacing consequitive co-directed vectors with their sum.

Firstly we should reorder the vertices in such a way that the first vertex
of each polygon has the lowest y-coordinate (in case of several such vertices pick the one with the smallest x-coordinate). After that the sides of both polygons
will become sorted by polar angle.
Now we create two pointers $i$ (pointing to a vertex of $P$) and $j$ (pointing to a vertex of $Q$), both initially set to 0.
Then while $i < |P|$ or $j < |Q|$:
\begin{enumerate}
\item Append $P_i + Q_j$ to $P + Q$.
\item Compare polar angles of $\overrightarrow{P_iP_{i + 1}}$ and $\overrightarrow{Q_jQ_{j+1}}$.
\item Increment the pointer which corresponds to the smallest angle (if the angles are equal, increment both).
\end{enumerate}

To find the distance between two convex polygons $P$ and $Q$ (or check whether they intersect), if we reflect $Q$ through the point $(0, 0)$ obtaining polygon $-Q$, the problem boils down to finding the smallest distance between a point in
$P + (-Q)$ and $(0, 0)$.

\subsubsection{Pythagorean Triples}
 The Pythagorean triples are uniquely generated by $$a=k\cdot (m^{2}-n^{2}),\ \,b=k\cdot (2mn),\ \,c=k\cdot (m^{2}+n^{2})$$
 with $m > n > 0$, $k > 0$, $m \bot n$, and either $m$ or $n$ even.

\subsubsection{Generare numeri primi}
%\begin{center}
\verb|openssl prime -generate -bits 60|
%\end{center}

\subsubsection{DP} Prova a pensarla sia in avanti che all’indietro, magari si semplifica qualcosa. (Es. i casi base se si va in avanti sono multipli, mentre
andando all’indietro è uno solo). Ricorda che esistono i bitset.
